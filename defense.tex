% arara: lualatex
\pdfoptionpdfminorversion 7
\documentclass{beamer}
\title{High Pressure Ignition Chemistry of Alternative Fuels}
\author{Bryan W. Weber}
\institute{Prepared for Ph.D. Defense}
\date{June 19, 2014}

\usepackage{lmodern}

\AtBeginSection{\frame{\sectionpage}}

\defbeamertemplate{section page}{mine}[1][]{%
  \begin{centering}
    {\usebeamerfont{section name}\usebeamercolor[fg]{section name}#1}
    \vskip1em\par
    \begin{beamercolorbox}[sep=12pt,center]{part title}
      \usebeamerfont{section title}\insertsection\par
    \end{beamercolorbox}
  \end{centering}
}
\setbeamertemplate{section page}[mine]

\usepackage{pgfplots}
\usetikzlibrary{decorations.pathreplacing,calc,decorations.text}

\newcommand{\arcarrow}[9]% inner radius, middle radius, outer radius, start angle, end angle, tip protusion angle, options, text, text color
{   \pgfmathsetmacro{\rin}{#1}
    \pgfmathsetmacro{\rmid}{#2}
    \pgfmathsetmacro{\rout}{#3}
    \pgfmathsetmacro{\astart}{#4}
    \pgfmathsetmacro{\aend}{#5}
    \pgfmathsetmacro{\atip}{#6}
    \fill[#7] (\astart:\rin) arc (\astart:\aend:\rin) -- (\aend+\atip:\rmid) -- (\aend:\rout) arc (\aend:\astart:\rout) -- (\astart+\atip:\rmid) -- cycle;
    \path[decoration={text along path, text={#8}, text color={#9}, text align={align=center}, raise=-0.5ex},decorate] (\astart+\atip:\rmid) arc (\astart+\atip:\aend+\atip:\rmid);
}

\newcommand{\tikzmark}[1]{\tikz[overlay,remember picture] \coordinate (#1);}

\beamertemplatenavigationsymbolsempty

\setbeamertemplate{footline}
{%
\begin{beamercolorbox}[sep=2mm]{}
\includegraphics[height=0.25in]{logo}
\hfill
{\color{gray} \insertpagenumber{}/\insertpresentationendpage}
\end{beamercolorbox}
}%

\graphicspath{ {figures/} }

\usepackage[absolute,overlay]{textpos}

\usepackage{siunitx}
% \usepackage[texcoord,grid,gridunit=mm,gridcolor=red!10,subgridcolor=green!10]{eso-pic}

\begin{document}

\maketitle

\begin{frame}<1-4>[label=economy]{Combustion drives the energy economy}
    \only<2>{
        \begin{center}
            \begin{tikzpicture}
    \begin{axis}[
        xtick={0,1,...,7},
        xticklabels={
            {Petroleum},
            {Natural Gas},
            {Coal},
            {Nuclear},
            {Hydropower},
            {Biomass},
            {Renewable Energy},
            {Other},
        },
        x tick label style={
            text height=2ex,
            rotate=45,
            anchor=north east,
            font=\footnotesize,
            yshift=10pt,
            align=right,
        },
        ybar,
        ylabel={Percent},
        title={United States Energy Consumption by Fuel Source},
        height=0.8\textheight,
        width=\textwidth,
        area legend,
    ]
        \pgfmathsetmacro\totaltwentythirteen{96.49}
        \addplot coordinates {
            (0,35.97/\totaltwentythirteen*100)
            (1,26.22/\totaltwentythirteen*100)
            (2,18.46/\totaltwentythirteen*100)
            (3,7.93/\totaltwentythirteen*100)
            (4,2.59/\totaltwentythirteen*100)
            (5,2.68/\totaltwentythirteen*100)
            (6,2.26/\totaltwentythirteen*100)
            (7,0.39/\totaltwentythirteen*100)
        };
        \pgfmathsetmacro\totaltwentyforty{106.31}
        \addplot coordinates {
            (0,35.35/\totaltwentyforty*100)
            (1,32.32/\totaltwentyforty*100)
            (2,18.75/\totaltwentyforty*100)
            (3,8.49/\totaltwentyforty*100)
            (4,2.90/\totaltwentyforty*100)
            (5,4.26/\totaltwentyforty*100)
            (6,3.89/\totaltwentyforty*100)
            (7,0.35/\totaltwentyforty*100)
        };
        \legend{2013,2040}
        \node[anchor=west,align=left] at (axis cs:3.5,20) {Data from EIA Annual\\Energy Report 2014};
    \end{axis}
\end{tikzpicture}

        \end{center}
    }
    \only<1,3->{
        \begin{itemize}
            \item Combustion is predicted to remain the dominant energy conversion process for many years into the future
            \item The combustion of fossil fuels has been implicated in a number of harmful effects on human health, the environment, and the economy
            \item Two solutions have been proposed:
                \begin{itemize}
                    \item<alert@4> Better engines
                    \item<alert@5> Better fuels
                \end{itemize}
        \end{itemize}
    }
\end{frame}

\begin{frame}{Better engines have higher efficiency and lower emissions}
    \begin{center}
        \includegraphics[width=\textwidth]{jackie-dec-engine}\\
        Reproduced in part from J.E. Dec, Proc. Combust. Inst. 32 (2009) 2727--2742
    \end{center}
\end{frame}

\againframe<5>{economy}

\begin{frame}<1>[label=betterfuels]{Better fuels have lower emissions and reduce dependence on fossil fuels}
    \setbeamercovered{transparent=15}
    \begin{itemize}
        \item<+- | uncover@+> Better fuels help move us away from traditional fuel sources
        \item<visible@+ | uncover@+> Better fuels help reduce total emissions
    \end{itemize}
\end{frame}

\begin{frame}{Methylcyclohexane is a major component in stepping-stone fuels}
    \begin{itemize}
        \item Methylcyclohexane (MCH) is an important component of fuels
            produced from alternative petroleum sources, such as shale oil
        \item Models of real transportation fuels are difficult to
            construct and use due to the chemical complexity of the fuels
        \item Surrogate models use a limited number of components to
            represent the chemical and physical properties of the real fuel
        \item MCH is a component in many surrogate transportation fuel
            formulations
    \end{itemize}
\end{frame}

\againframe<2>{betterfuels}

\begin{frame}{Bio-alcohol fuels can be sustainably produced to reduce well-to-wheel emissions}
    \begin{columns}
        \column{0.5\textwidth}
            \begin{itemize}
                \item Ethanol is a very common biofuel in use today
                \item Alcohols with more carbon atoms have higher energy density
                \item Butanol and pentanol can be produced from bio-based and waste sources
                \item Butanol is the smallest alcohol with each type of C-O bond
            \end{itemize}
        \column{0.5\textwidth}
            \centering
            \includegraphics[height=0.18\textheight]{1-butanol}\par
            \includegraphics[height=0.18\textheight]{2-butanol}\par
            \includegraphics[height=0.18\textheight]{t-butanol}\par
            \includegraphics[height=0.18\textheight]{iso-butanol}\par
            \includegraphics[height=0.18\textheight]{ipentanol}
    \end{columns}
\end{frame}

\begin{frame}{We need both solutions to make substantial progress}
    \begin{columns}
        \column{0.5\textwidth}
            \begin{itemize}
                \item<+-> Selecting the ``best'' alternative fuel requires knowledge
                    of the ``best'' engine, which depends on which
                    alternative fuel is selected\ldots
                \item<.-> Testing every fuel in every engine is prohibitively expensive
                    and time consuming
                \item<+-> Computer-aided design can be employed to create fuel-flexible
                    engines \alert{if the fuel models are predictive under LTC conditions}
            \end{itemize}
        \column{0.5\textwidth}
            \centering
            \only<2>{\setbeamercovered{transparent=15}}
            \begin{tikzpicture}[scale=0.6]
                \uncover<1>{\fill[even odd rule,red!30] circle (3.8) circle (3.2);
                \arcarrow{3}{3.5}{4}{-80}{80}{5}{blue,draw=red!50!black,very thick}{Best Advanced Engine}{white}
                \arcarrow{3}{3.5}{4}{100}{260}{5}{blue,draw=red!50!black,very thick}{Best Alternative Fuel}{white}}
                \node<2> at (0,0) {\includegraphics[width=0.87\textwidth]{computer}};
            \end{tikzpicture}
    \end{columns}
\end{frame}

\begin{frame}
    \begin{center}
    \begin{tikzpicture}
    \begin{semilogyaxis}[
        table/col sep=comma,
        ylabel style={align=center,yshift=-5pt},
        xlabel style={align=center,yshift=5},
        ylabel={Reactivity\\$\leftarrow$Higher\qquad\qquad Lower$\rightarrow$},
        xlabel={$\leftarrow$Higher\qquad\qquad Lower$\rightarrow$\\Temperature},
        yticklabels={,,},
        xticklabels={,,},
        xmin=0.7,xmax=1.5,
        ymin=0.02,ymax=200,
        tick style = {color=black, semithick},
        minor tick style={thin},
        axis line style={semithick},
        every axis plot/.append style={%
            semithick,
        },
        every axis legend/.append style={
            legend pos=south east,
            anchor=south east,
            font=\tiny,
        },
        legend columns=1,
    ]
    \only<1-6>{\addplot[only marks,mark=square*,red] table[x=Hi-T-Real-Tc, y=Hi-T-Real-Ign] {figures/alcohol-models/model-data.csv};
    \addlegendentry{High Temperature Alkane}}
    \only<2-6>{\addplot[no markers,cyan] table[x=Hi-T-Alc-Tc, y expr=\thisrow{Hi-T-Alc-Mod}*0.75] {figures/alcohol-models/model-data.csv};
    \addlegendentry{High Temperature Model}}
    \only<3-6>{\addplot[only marks,mark=diamond*,black] table[x=Lo-T-Real-Tc, y=Lo-T-Real-Ign] {figures/alcohol-models/model-data.csv};
    \addlegendentry{Low Temperature Alkane}}
    \only<4-6>{\addplot[no markers,orange] table[x=Real-Mod-Tc, y=Real-Mod-Ign] {figures/alcohol-models/model-data.csv};
    \addlegendentry{Comprehensive Alkane Model}}
    \only<5->{\addplot[mark=*,blue,only marks,restrict x to domain=0.7:1.15] table[x=Alcohol-Tc, y=Alcohol-Ign] {figures/alcohol-models/model-data.csv};
    \addlegendentry{High Temperature Alcohol}}
    \only<6->{\addplot[only marks,mark=oplus*,violet,restrict x to domain=1.15:1.5] table[x=Alcohol-Tc, y=Alcohol-Ign] {figures/alcohol-models/model-data.csv};
    \addlegendentry{Low Temperature Alcohol}}
    \only<7>{\addplot[no markers,green!40!black,] table[x={1000/T_(1/K)},y expr=1000*\thisrow{Ignition time 3 by max dT/dt_(sec)}] {figures/alcohol-models/export.csv};
    \addlegendentry{Alcohol by Analogy to Alkane}}
    \only<8->{\addplot[no markers,domain=0.8:1.4,gray] {1.4097*x^12.714};
    \addlegendentry{Comprehensive Alcohol Model}}
    \end{semilogyaxis}
    \end{tikzpicture}
    \end{center}
\end{frame}

\begin{frame}{How can we understand the effect of better fuels on better engines?}
    \begin{itemize}
        \item We need to know the physical properties
        \begin{itemize}
            \item Density
            \item Viscosity
            \item \ldots
        \end{itemize}
        \item We need to know the combustion properties
        \begin{itemize}
            \item Heat of combustion
            \item<alert@2> Reactivity \only<2->{$\leftarrow$ Ignition Delay}
            \item<alert@3> Pollutant Production \only<3->{$\leftarrow$ Detailed reaction pathways}
            \item \ldots
        \end{itemize}
    \end{itemize}
\end{frame}

\section{Experimental Methods}

\begin{frame}{Rapid Compression Machine}
    \begin{center}
        \includegraphics<+>[width=\textwidth]{rcm-photo}
        \includegraphics<+>[height=0.95\textheight]{ign-delay-def}
    \end{center}
\end{frame}

\begin{frame}{Rapid Sampling Apparatus}
    \begin{columns}
        \column{0.4\textwidth}
            \begin{itemize}[<+->]
                \item Sampling apparatuses have been used since the 1920's to study combustion chemistry
                \item In the 1960's, the first sampling apparatus was adapted for an RCM
                \item Sampling devices rapidly quench ongoing reactions so that species are determined at a discrete point in time
            \end{itemize}
        \column{0.6\textwidth}
            \centering
            \begin{minipage}[c][0.95\textheight][c]{\textwidth}
            \only<1>{
                \includegraphics[height=0.95\textheight]{brooks}
            }
            \only<2,4>{
                \includegraphics[width=\textwidth]{roblee-sampling}
            }
            \only<3>{
                \includegraphics[width=\textwidth]{rsa-pressure}
            }
            \end{minipage}
    \end{columns}
\end{frame}

\begin{frame}{Rapid Compression Machine}
    \begin{center}
        \includegraphics<+>[width=\textwidth]{rcm-with-tank}
        \includegraphics<+>[height=0.85\textheight]{rcm-schematic}
    \end{center}
\end{frame}

\begin{frame}{Gas Chromatograph/Mass Spectrometer}
    \begin{columns}
        \column{0.4\textwidth}
            \begin{itemize}[<only@+>]
                \item Standard piece of chemistry lab equipment, commercially supplied (Shimadzu GCMS-QP2010S)
                \item Separates, identifies, and quantifies chemical species
            \end{itemize}
        \column{0.6\textwidth}
            \centering
            \includegraphics<1>[width=\textwidth]{gcms-photo}
            \includegraphics<2>[height=0.45\textheight]{gcms-buoh}\par
            \includegraphics<2>[height=0.45\textheight]{mch-tic}
    \end{columns}
\end{frame}

\begin{frame}<1,2>[label=numerical]{Numerical Methods}
    \begin{itemize}
        \item Computational analysis complements experimental work
        \item Direct comparisons can be made between the computed and experimental
            ignition delays as well as the pressure traces
        \item Detailed analysis of chemical kinetic models can deepen the
            understanding of the experimental data
            \begin{itemize}
                \item<+(1)- | alert@+(1)> Path Flux Analysis helps determine important reaction pathways
                \item<+(1)- | alert@+(1)> Sensitivity Analysis helps find important reactions and how they affect reactivity
            \end{itemize}
    \end{itemize}
\end{frame}

\begin{frame}{Path Flux Analysis}
    \begin{center}
        \includegraphics[width=\textwidth]{nbuoh-first-products}
    \end{center}
\end{frame}

\againframe<3>{numerical}

\begin{frame}{Sensitivity Analysis}
    \begin{center}
        \includegraphics[height=0.8\textheight]{buoh-sens}
    \end{center}
\end{frame}

\section{Methylcyclohexane}

\begin{frame}{Ignition delays were measured at new conditions}
    \begin{center}
        \includegraphics[height=0.85\textheight]{mch-expts}
    \end{center}
\end{frame}

\begin{frame}{Many improvements to the kinetic model for MCH were made}
    \only<1,3,5>{
    \setbeamercovered{transparent=15}
    \begin{itemize}
        \item<1> The C$_1$--C$_4$, aromatics, and cyclohexane chemistry have been updated
        \item<1> Important fuel decomposition reaction rates have been updated based on
            experimental measurements and quantum chemical calculations
        \item<1> These changes have improved prediction of the overall ignition
            delay substantially
        \item<uncover@3 | invisible@1> The activation energy of an important class of decomposition
            reactions of MCH was increased to match the value used in the base chemistry
        \item<visible@5> New pathways involving species with unsaturated rings were added
            to enable prediction of species such as methylcyclohexenes
    \end{itemize}
    }
    \only<2>{
        \begin{columns}
                \column{0.5\textwidth}
                    \centering
                    \includegraphics[width=\textwidth]{mittal-mch}\par
                    Mittal and Sung Combust.\ Flame 2009
                \column{0.5\textwidth}
                    \centering
                    \includegraphics[width=\textwidth]{mch-model-1/mch-model-1}\par
                    Weber et al.\ Combust.\ Flame 2014
        \end{columns}
    }
    \only<4>{
        \begin{center}
            \includegraphics[height=0.85\textheight]{mch-energy}
        \end{center}
    }
\end{frame}

\begin{frame}{Path flux analysis shows the important reaction pathways}
    \begin{center}
        \begin{tikzpicture}
        \node at (0,0) {\includegraphics[height=0.8\textheight]{mch-path}};
        \begin{scope}[ultra thick,red]
        \draw<2> (-3.25,-3.35) rectangle +(0.75,1.1);
        \draw<2> (-0.1,-3) rectangle +(1.1,1);
        \draw<2> (1.9,-3.15) rectangle +(1.2,0.75);
        \draw<2> (4.15,2.5) rectangle +(-1.2,-0.75);
        \draw<2> (-4.15,2.5) rectangle +(1.2,-0.75);
        \end{scope}
        \end{tikzpicture}
    \end{center}
\end{frame}

\begin{frame}{New sampling data can test the newly added pathways}
    \begin{center}
        \includegraphics<1>[height=0.95\textheight]{rsa-pressure}
        \only<2-6>{
        \begin{columns}[T]
            \column{0.8\textwidth}
                \includegraphics<2>[height=0.95\textheight]{mch-tic/mch-tic-1}
                \includegraphics<3>[height=0.95\textheight]{mch-tic/mch-tic-2}
                \includegraphics<4>[height=0.95\textheight]{mch-tic/mch-tic-3}
                \includegraphics<5>[height=0.95\textheight]{mch-tic/mch-tic-4}
                \includegraphics<6>[height=0.95\textheight]{mch-tic/mch-tic-5}
            \column{0.2\textwidth}
                \only<2-3>{\tikzmark{mch}\includegraphics[height=0.15\textheight]{mch-skeletal}\par}
                \only<3>{\tikzmark{mchene-2}\includegraphics[height=0.15\textheight]{mchene-2}\par}
                \only<3>{\tikzmark{mchene-3}\includegraphics[height=0.15\textheight]{mchene-3}\par}
                \only<3>{\tikzmark{menech}\includegraphics[height=0.15\textheight]{menech}\par}
                \only<3>{\tikzmark{mchene-1}\includegraphics[height=0.15\textheight]{mchene-1}}
                \only<4>{\tikzmark{2-propenal}\includegraphics[height=0.15\textheight]{2-propenal}\par}
                \only<4>{\tikzmark{mac}\includegraphics[height=0.15\textheight]{methacrolein}\par}
                \only<4>{\tikzmark{mvk}\includegraphics[height=0.15\textheight]{methyl-vinyl-ketone}}
        \end{columns}
        \begin{tikzpicture}[overlay,remember picture,->,ultra thick]
        \draw<2-3> ($(mch)+(0,0.5)$) --     (-0.05,6.75);
        \draw<3> ($(mchene-2)+(0,0.25)$) -- (0.45,2.9);
        \draw<3> ($(mchene-3)+(0,0.25)$) -- (0.55,2.9);
        \draw<3> ($(menech)+(0,0.25)$) --   (0.65,1.75);
        \draw<3> ($(mchene-1)+(0,0.5)$) -- (2,2);
        \draw<4> ($(2-propenal)$) -- (-4,2.75);
        \draw<4> ($(mac)$) -- (-3.5,2);
        \draw<4> ($(mvk)$) -- (-3.35,2);
        \end{tikzpicture}
        }
        \includegraphics<7>[height=0.95\textheight]{mch-sim-pres-comp}
        \only<8>{Methylcyclohexane}\par\includegraphics<8>[height=0.85\textheight]{mch-mole-fracs}
        \only<9>{Methylcyclohexenes}\par\includegraphics<9>[height=0.85\textheight]{mchene-mole-fracs}
    \end{center}
\end{frame}

\section{Butanol Isomers}

\begin{frame}{The reactivity of the butanol isomers depends on the pressure}
    \only<1>{
        \begin{columns}
            \column{0.5\textwidth}
                \includegraphics[width=\textwidth]{buoh-15bar}
            \column{0.5\textwidth}
                \includegraphics[width=\textwidth]{buoh-30bar}
        \end{columns}
    }
    \begin{center}
        \includegraphics<2>[height=0.85\textheight]{tbuoh-phi10}
        \includegraphics<3>[height=0.85\textheight]{sbuoh-pressure/sbuoh-pressure}
    \end{center}
\end{frame}

\begin{frame}{Understanding the detailed chemistry enables modeling of biofuel ignition}
    \only<1>{
        \begin{center}
            \includegraphics<1>[height=0.85\textheight]{buoh-30sim}\\
            Model from Sarathy et al.\ Combust. Flame 2012
        \end{center}
    }
    \only<2>{
        \begin{columns}
            \column{0.5\textwidth}
                \includegraphics<2>[width=\textwidth]{tbuoh-sims}
            \column{0.5\textwidth}
                \includegraphics<2>[width=\textwidth]{tbuoh-20press}
        \end{columns}
    }
    \only<3-4>{
        \begin{center}
            \includegraphics<3>[height=0.8\textheight]{buoh-heat-inset}
            \includegraphics<4>[width=\textwidth]{buoh-heat-reactions}
        \end{center}
        \only<4>{
            \begin{textblock*}{50mm}(70mm,72.5mm)
                {\tiny Note: the reaction pictured here is actually the second most important reaction for \textit{n}-butanol}
            \end{textblock*}
        }
    }
\end{frame}

\begin{frame}{The models cannot predict the combustion behavior at some LTC engine-relevant conditions}
    \begin{columns}
        \column{0.45\textwidth}
            \centering
            \includegraphics[width=\textwidth]{ibuoh-30o2sim}\\
            Model from Weber et al.\ 8th US National Combustion Meeting 2013
        \column{0.45\textwidth}
            \centering
            \includegraphics[width=\textwidth]{ibuoh-30o2sim-sar}\\
            Model from Sarathy et al.\ Combust.\ Flame 2012
    \end{columns}
    \vspace{0.5cm}
    \begin{center}
        There is still some critical information missing from our
        understanding of high-pressure, low-temperature ignition
        of alternative fuels
    \end{center}
\end{frame}

\begin{frame}{Future work}
    \only<1>{
        \begin{itemize}
            \item Calibration for more species of importance
            \item Provide quantitative speciation data for model comparison
            \item Consider more conditions and more fuels
        \end{itemize}
    }
\end{frame}

\begin{frame}{Summary}
    \begin{itemize}
        \item We need a better understanding of the fundamental
            combustion properties of alternative fuels, especially under
            engine-relevant conditions
        \item Using this understanding, we need to develop models that
            can predict the combustion behavior of new fuels in new engines
        \item Through experimental studies and detailed chemical kinetic
            analysis, my work has extended the state-of-the-art in our
            understanding of high-pressure ignition chemistry of alternative
            fuels
    \end{itemize}
\end{frame}

\end{document}
